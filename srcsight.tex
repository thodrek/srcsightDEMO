% THIS IS AN EXAMPLE DOCUMENT FOR VLDB 2012
% based on ACM SIGPROC-SP.TEX VERSION 2.7
% Modified by  Gerald Weber <gerald@cs.auckland.ac.nz>
% Removed the requirement to include *bbl file in here. (AhmetSacan, Sep2012)
% Fixed the equation on page 3 to prevent line overflow. (AhmetSacan, Sep2012)

\documentclass{vldb}
\usepackage{graphicx}
\usepackage{balance}  % for  \balance command ON LAST PAGE  (only there!)
\usepackage{cleveref}

\newcommand\system{\textsc{SourceSight}}


\begin{document}

% ****************** TITLE ****************************************

\title{{\LARGE \system}: Discovering Valuable Sources for Integration}

\numberofauthors{5} 

\author{
% 1st. author
\alignauthor
Theodoros Rekatsinas\\
       \affaddr{University of Maryland}\\
       \email{thodrek@cs.umd.edu}
% 2nd. author
\alignauthor
Amol Deshpande\\
       \affaddr{University of Maryland}\\
       \email{amol@cs.umd.edu}
% 3rd. author
\alignauthor 
Xin Luna Dong\\
       \affaddr{Google Inc.}\\
       \email{lunadong@google.com}
\and  % use '\and' if you need 'another row' of author names
% 4th. author
\alignauthor 
Lise Getoor\\
       \affaddr{UC Santa Cruz}\\
       \email{getoor@soe.ucsc.edu}
% 5th. author
\alignauthor Divesh Srivastava\\
       \affaddr{AT\& T Labs-Research}\\
       \email{divesh@research.att.com}
}

\maketitle

\begin{abstract}
Lately there has been a rapid increase in the number of data sources and public access data services, such as cloud-based data markets and data portals, that facilitate the collection, publishing and trading of data. Data sources typically exhibit large heterogeneity in the type and quality of data they provide. Unfortunately, when the number of data sources is large, humans have a limited capability of reasoning about the actual usefulness of sources and the trade-offs between the benefits and costs of acquiring and integrating sources. In this demonstration we present \system , a framework that facilitates the exploration and selection of sources for integration. \system is based on the intuition that different sets of data sources are optimal to integrate for different integration tasks. The main focus of \system is to allow users to discover and explore valuable sets of sources for diverse integration tasks over real-world event-data sources.
\end{abstract}


\section{Introduction}
Data integration and data cleaning remain among the most cost-intensive tasks in data management, either due to the increased computation and human-labor costs involved in the process~\cite{kruse2015estimating} or the monetary cost involved in acquiring data from different sources~\cite{balazinska:vldb11}. Given the high number of available data sources and the aforementioned costs, it is challenging for a user to identify the sources that are truly beneficial to her application. This gives rise to the natural question of how can one discover {\em the most beneficial sources} for integration, i.e., sources that maximize the user's benefit at the minimum cost. 

Several approaches have been proposed to help users reason about the value of integrating multiple data sources. The proposed approaches can be divided in two main categories: (i) {\em effort-oriented} techniques that focus on characterizing solely the cost of integration, and (ii) data-oriented techniques that reason about the benefit of integration. Techniques from the first category estimate the cost of integration via reasoning about the effort required to perform schema-matching, data-cleaning and data-transformations when integrating multiple sources~\cite{kruse2015estimating, smith:2009}. The main limitation of th Here, the sources are {\em fixed}, assuming that the user has already specified the sources that are relevant to her query and given the estimated integration cost 

describe \system and its unique contributions, i.e., functionalities for the users.

organization of the demo

\section{Selecting Sources for Integration}

analyze the content of sources, identify the domains that sources cover and materialize quality profiles. 

Different quality metrics available 

Users provide a query that specifies a domain

Oracle provides the quality profiles of sources for the specified domain

Perform source selection

\section{Overview of the System}

Describe the system architecture to perform these operations

\section{Demo Details}

Explore correspondence graph to help users identify the domain of their interest

Get an initial understanding of what's the quality of relevant sources

Perform source selection and identify sets of sources that optimize for different
quality metrics

Compare user constructed solutions with source-selection solutions

Allow users to remove and add sources

\bibliographystyle{abbrv}
\bibliography{srcsight}


\end{document}
